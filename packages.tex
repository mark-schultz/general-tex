
\usepackage{etex}                         % Allows more ``variables'' saved.
\usepackage{amsmath,amssymb}       % Standard AMS packages
\usepackage{accents}			                % \accentset{accent}{symbol} - Like \stackrel
\usepackage{mathtools}										% \intertext{},\shortintertext{},\Aboxed{a&b}, extensible arrows
\usepackage{mathrsfs} 	                  % Topology T
\usepackage{empheq}                       % Allows modifications to most environments (like align).
\usepackage{enumitem}			                % For \begin{enumerate}[resume] and [start = 3]
%\usepackage{gitinfo}		                  % \gitAuthorName, \gitAuthorEmail, \gitAuthorDate
\usepackage{mdframed}		                  % For nice question environment
\usepackage{tikz}			                    % Same as above
\usepackage{etoolbox}
\usepackage[utf8]{inputenc}
\usepackage[T1]{fontenc}
\usepackage{lmodern}
\usepackage{centernot}	%%gives \centernot, better than \not
\usepackage{tikz-cd}

\allowdisplaybreaks % Breaks align environments across pages
%%%%%%%%%%%%%%%% Mark's Macros %%%%%%%%%%%%%%%%%%%%%%%%%%%%%%
%%
%% Greek Characters
\newcommand{\Ga}{\Gamma}
\newcommand{\ga}{\gamma}
\newcommand{\e}{\varepsilon}
\newcommand{\w}{\omega}
\newcommand{\W}{\Omega}

%% Fraktur Characters
\newcommand{\mm}{\mathfrak{m}}
\newcommand{\oldaa}{\aa}			%% accent a with a circle on top
\renewcommand{\aa}{\mathfrak{a}}
\newcommand{\bb}{\mathfrak{b}}
%% Fields/Vector Spaces
\renewcommand{\AA}{\mathbb{A}}
\newcommand{\CC}{\mathbb{C}}
\newcommand{\FF}{\mathbb{F}}
\newcommand{\PP}{\mathbb{P}}
\newcommand{\QQ}{\mathbb{Q}}
\newcommand{\ZZ}{\mathbb{Z}}
\newcommand{\NN}{\mathbb{N}}
\newcommand{\TT}{\mathbb{T}}
\newcommand{\RR}{\mathbb{R}}
\newcommand{\RRe}{\RR\cup\Bqty{\infty}}
\newcommand{\RRE}{\RR\cup\Bqty{\pm\infty}}
\newcommand{\EE}{\mathbb{E}}
\newcommand{\HH}{\mathbb{H}} %% Complex Upper-Half Plane
\newcommand{\BB}{\mathbb{B}} %% (open) Ball
\newcommand{\GG}{\mathbb{G}}
%%
%% Matrix Operators
\DeclareMathOperator{\End}{End}
\DeclareMathOperator{\id}{id}
\DeclareMathOperator{\im}{im}
\newcommand{\matx}[1]{\bigl(\begin{smallmatrix} #1 \end{smallmatrix}\bigr)}
\newcommand{\Matx}[1]{\begin{pmatrix}#1\end{pmatrix}}
\DeclareMathOperator{\rk}{rk}
\DeclareMathOperator{\tr}{tr}
\DeclareMathOperator{\Gen}{Gen}
\DeclareMathOperator{\Enc}{Enc}
\DeclareMathOperator{\Dec}{Dec}
\DeclareMathOperator{\Priv}{Priv}
\DeclareMathOperator{\ext}{Ext}
\DeclareMathOperator{\Int}{Int}
\DeclareMathOperator{\negl}{negl}
\DeclareMathOperator{\Mac}{Mac}
\DeclareMathOperator{\Vrfy}{Vrfy}
\newcommand{\bin}{\{0,1\}}
\newcommand{\bqty}[1]{\{ #1\}}
%%
%% Category Theory stuff
\newcommand{\catformat}[2]{\ensuremath{\mathfrak{#1}}\text{#2} }
\newcommand{\set}{\catformat{S}{et}}
\newcommand{\oldtop}{\top}		% Flipped \perp
\renewcommand{\top}{\catformat{T}{op}}
\newcommand{\man}{\catformat{M}{an}}
\newcommand{\grp}{\catformat{G}{rp}}
\newcommand{\abgrp}{\ensuremath{\mathfrak{A}}\text{b}\grp}
\newcommand{\vect}{\catformat{V}{ect}}
\newcommand{\metric}{\catformat{M}{etric}}
\let\hom\relax		% AMSmath defines \hom similar to \log
\DeclareMathOperator{\hom}{Hom}
\DeclareMathOperator{\colim}{colim}
\newcommand{\radjoint}{\vdash}		% Right adjoint
\newcommand{\ladjoint}{\dashv}		% Left adjoint
\newcommand{\opensub}{\stackrel{\text{open}}{\subseteq}}
\newcommand{\closedsub}{\stackrel{\text{closed}}{\subseteq}}
\newcommand{\actson}{\circlearrowright} % For topological group actions


%% Algebraic Structures
\renewcommand{\H}{\mathcal{H}}  % old \H{x} is an x with a weird umlaut in text mode
\newcommand{\M}{\mathcal{M}}
\newcommand{\K}{\mathcal{K}}
\newcommand{\mou}[1]{\ensuremath{\operatorname*{#1}}} %Math operator, limit underneath
\newcommand{\mo}[1]{\ensuremath{\operatorname{#1}}} %Math operator normal
\newcommand{\tee}{\mathscr{T}}  % Topology T
\newcommand{\bee}{\mathscr{B}}	% Topology B for basis
\newcommand{\cl}[1]{\overline{#1}}
\newcommand{\ii}{\mathbf{i}}		% Identity arithmetic function
\newcommand{\B}{\mathcal{B}}  	% Borel Algebra
\newcommand{\oldL}{\L}          % Some accented character
\renewcommand{\L}{\mathcal{L}}	% Lesbegue Algebra
\newcommand{\C}{\mathcal{C}}    % Cantor Set
\newcommand{\I}{\mathcal{I}}		% Lebesgue Integrable functions.
\newcommand{\A}{\mathcal{A}}	  % Functions that separate points
\newcommand{\oldO}{\O} 					% 0 with line through it, ``fat emptyset''
\renewcommand{\O}{\mathcal{O}}	% Integer ring of number field
\newcommand{\h}{\mathfrak{h}}

\newcommand{\surface}[2]{\langle #1\mid#2\rangle}  %Topology surface presentation
%%
%% Math Operators
\DeclareMathOperator{\aut}{Aut}
\DeclareMathOperator{\GL}{GL}
\DeclareMathOperator{\Mat}{Mat}
\DeclareMathOperator{\SL}{SL}
\DeclareMathOperator{\Bl}{Bl}
\DeclareMathOperator{\Frac}{Frac}
\newcommand{\ber}[1]{\mo{Ber}\pqty{#1}}
\newcommand{\Binom}[2]{\mo{Binom}\pqty{#1,#2}}
\DeclareMathOperator{\diam}{diam}
%%
%% Legendre Symbol
\newcommand{\leg}[2]{\left(\frac{#1}{#2}\right)}
%% Misc. Elementary Things
\newcommand{\dd}{\,\mathrm{d}}	% Integral dx
\def\<{\langle}
\def\>{\rangle}
\newcommand\der[2]{\frac{d #1}{d #2}}		% Derivative
\newcommand\pder[2]{\frac{\partial #1}{\partial #2}} % Partial derivative
\newcommand{\inv}{^{-1}}
\let\existstemp\exists
\let\foralltemp\forall
\renewcommand*{\exists}{\existstemp\mkern2mu}
\renewcommand*{\forall}{\foralltemp\mkern2mu} %%Better quantifier spacing
%%
%% Shorthand Stuff
\newcommand{\Mark}[1]{[[\ensuremath{\bigstar\bigstar\bigstar} #1]]} %Sign to revisit after lecture
\newcommand{\hhat}[1]{\widehat{#1}}
\renewcommand{\labelitemi}{--}  % changes the default bullet in itemize
\newcommand{\pqty}[1]{\left( #1 \right)} %Resizable Parenthesis
\newcommand{\Bqty}[1]{\left\lbrace #1\right\rbrace} %Resizable Braces
\newcommand{\vqty}[1]{\left\lVert #1 \right\rVert} %Norm symbol
\newcommand{\abs}[1]{\left|#1\right|}
\newcommand{\ip}[2]{\left\langle #1,#2\right\rangle}

\newcommand{\Def}[1]{\textbf{#1}\index{#1}}
\newcommand{\lec}[1]{\begin{center}\textbf{LECTURE - #1}\end{center}}
\newcommand{\URL}[1]{\href{#1}{\url{#1}}}	%% Displays and links to url
%%%%%%%%%%%% End Mark's Macros %%%%%%%%%%%%%%%%%%%%%%%%%%%%%%%%%%%%%%%

\author{Mark Schultz}

%%%%%%%%%%% Box around Questions on Problem Sets %%%%%%%%%%%%%%%%%%%%%

\newcounter{Problem}
\newcounter{SubProblem}[Problem]

\global\mdfdefinestyle{problem}{%
	linecolor=lightgray,linewidth=1pt,%
	leftmargin=0cm,rightmargin=0cm,
}
\global\mdfdefinestyle{subproblem}{%
	linecolor=lightgray,linewidth=1pt,%
	leftmargin=.3cm,rightmargin=0cm,
}


\newenvironment{problem}[1][]{\refstepcounter{Problem}\par%
	\mdfsetup{%
		frametitle={\tikz\node[fill=white,rectangle,inner sep=0pt,outer sep=0pt]{Problem \arabic{Problem}};},
		frametitleaboveskip=-0.5\ht\strutbox,
		frametitlealignment=\raggedright
	}%
	\begin{mdframed}[style=problem]
		}{\end{mdframed}}

	\newenvironment{subproblem}[1][]{\refstepcounter{SubProblem}\par%
		\mdfsetup{%
			frametitle={\tikz\node[fill=white,rectangle,inner sep=0pt,outer sep=0pt]{Part \alph{SubProblem}};},
			frametitleaboveskip=-0.5\ht\strutbox,
			frametitlealignment=\raggedright,leftmargin=.5cm
		}%
		\begin{mdframed}[style=subproblem]}{\end{mdframed}}

	\endinput
