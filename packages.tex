
\usepackage{etex}                         % Allows more ``variables'' saved.
\usepackage{amsmath,amssymb}              % Standard AMS packages
\usepackage{accents}			                % \accentset{accent}{symbol} - Like \stackrel
\usepackage{mathtools}										% \intertext{},\shortintertext{},\Aboxed{a&b}, extensible arrows
\usepackage{mathrsfs} 	                  % Topology T
\usepackage{empheq}                       % Allows modifications to most environments (like align).
\usepackage{enumitem}			                % For \begin{enumerate}[resume] and [start = 3]
%\usepackage{gitinfo}		                  % \gitAuthorName, \gitAuthorEmail, \gitAuthorDate
\usepackage{mdframed}		                  % For nice question environment
\usepackage{tikz}			                    % Same as above
\usepackage{etoolbox}
\usepackage[utf8]{inputenc}
\usepackage[T1]{fontenc}
\usepackage{lmodern}
\usepackage{centernot}	%%gives \centernot, better than \not
\usepackage{tikz-cd}

\allowdisplaybreaks % Breaks align environments across pages
%%%%%%%%%%%%%%%% Mark's Macros %%%%%%%%%%%%%%%%%%%%%%%%%%%%%%

%% Fields/Vector Spaces
\newcommand{\A}{\mathbb{A}}
\newcommand{\B}{\mathbb{B}} %% (open) Ball
\newcommand{\C}{\mathbb{C}}
\newcommand{\D}{\mathbb{D}}
\newcommand{\E}{\mathbb{E}}
\newcommand{\F}{\mathbb{F}}
\newcommand{\G}{\mathbb{G}}
\renewcommand{\H}{\mathbb{H}} %% Double end quotes I think
\newcommand{\N}{\mathbb{N}}
\renewcommand{\P}{\mathbb{P}}	%% Old is paragraph marking
\newcommand{\Q}{\mathbb{Q}}
\newcommand{\R}{\mathbb{R}}
\newcommand{\T}{\mathbb{T}}
\newcommand{\Z}{\mathbb{Z}}
%%
%% Matrix Operators
\DeclareMathOperator{\End}{End}
\DeclareMathOperator{\id}{id}
\DeclareMathOperator{\im}{im}
\DeclareMathOperator{\rk}{rk}
\DeclareMathOperator{\tr}{tr}


%% CS Stuff
\newcommand{\bin}{\{0,1\}}

%% Paired delimiters
\newcommand{\matx}[1]{\bigl(\begin{smallmatrix} #1 \end{smallmatrix}\bigr)}
\newcommand{\Matx}[1]{\begin{pmatrix}#1\end{pmatrix}}
\newcommand{\bqty}[1]{\{ #1\}}
\newcommand{\pqty}[1]{\left( #1 \right)} %Resizable Parenthesis
\newcommand{\Bqty}[1]{\left\lbrace #1\right\rbrace} %Resizable Braces
\newcommand{\vqty}[1]{\left\lVert #1 \right\rVert} %Norm symbol
\newcommand{\abs}[1]{\left|#1\right|}
\newcommand{\ip}[2]{\left\langle #1,#2\right\rangle}
%%

%% Quick math operators
\newcommand{\mo}[1]{\ensuremath{\operatorname{#1}}} %Math operator normal
\newcommand{\mou}[1]{\ensuremath{\operatorname*{#1}}} %Math operator, limit underneath


%% Misc Elementary Things
\newcommand{\cl}[1]{\overline{#1}}
\newcommand{\dd}{\,\mathrm{d}}	% Integral dx
\def\<{\langle}
\def\>{\rangle}
\newcommand\der[2]{\frac{d #1}{d #2}}		% Derivative
\newcommand\pder[2]{\frac{\partial #1}{\partial #2}} % Partial derivative
\newcommand{\inv}{^{-1}}
\let\existstemp\exists
\let\foralltemp\forall
\renewcommand*{\exists}{\existstemp\mkern2mu}
\renewcommand*{\forall}{\foralltemp\mkern2mu} %%Better quantifier spacing

%% Shorthand Stuff
\newcommand{\Mark}[1]{[[\ensuremath{\bigstar\bigstar\bigstar} #1]]} %Sign to revisit after lecture
\renewcommand{\labelitemi}{--}  % changes the default bullet in itemize
\newcommand{\Def}[1]{\textbf{#1}\index{#1}}
\newcommand{\lec}[1]{\begin{center}\textbf{LECTURE - #1}\end{center}}
\newcommand{\URL}[1]{\href{#1}{\url{#1}}}	%% Displays and links to url
%%%%%%%%%%%% End Mark's Macros %%%%%%%%%%%%%%%%%%%%%%%%%%%%%%%%%%%%%%%

\author{Mark Schultz}

%%%%%%%%%%% Box around Questions on Problem Sets %%%%%%%%%%%%%%%%%%%%%

\newcounter{Problem}
\newcounter{SubProblem}[Problem]

\global\mdfdefinestyle{problem}{%
	linecolor=lightgray,linewidth=1pt,%
	leftmargin=0cm,rightmargin=0cm,
}
\global\mdfdefinestyle{subproblem}{%
	linecolor=lightgray,linewidth=1pt,%
	leftmargin=.3cm,rightmargin=0cm,
}


\newenvironment{problem}[1][]{\refstepcounter{Problem}\par%
	\mdfsetup{%
		frametitle={\tikz\node[fill=white,rectangle,inner sep=0pt,outer sep=0pt]{Problem \arabic{Problem}};},
		frametitleaboveskip=-0.5\ht\strutbox,
		frametitlealignment=\raggedright
	}%
	\begin{mdframed}[style=problem]
		}{\end{mdframed}}

	\newenvironment{subproblem}[1][]{\refstepcounter{SubProblem}\par%
		\mdfsetup{%
			frametitle={\tikz\node[fill=white,rectangle,inner sep=0pt,outer sep=0pt]{Part \alph{SubProblem}};},
			frametitleaboveskip=-0.5\ht\strutbox,
			frametitlealignment=\raggedright,leftmargin=.5cm
		}%
		\begin{mdframed}[style=subproblem]}{\end{mdframed}}

	\endinput
